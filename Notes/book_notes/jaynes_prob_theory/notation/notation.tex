\documentclass[../jaynes_prob_theory_notes.tex]{subfiles}
%\usepackage{amsmath}
%\usepackage[margin=1in]{geometry}

\begin{document}

\section{Notation and Equations}

\subsection{Notation}
    \begin{itemize}
        \item Greek letters ($\alpha$, $\beta$, etc.) denote continuously variable parameters and Latin letters (a, b, etc) denote discrete indices or data values, unless otherwise stated
        \item Probabilities are denoted by capital P's, which signifies that the arguments are propositions
            \begin{itemize}
                \item[] $P(A|B)$
            \end{itemize}
    
        \item Probabilities whose arguments are numerical values are denoted by other function symbols, such as
            \begin{itemize}
                \item[] $f(r|np)$
            \end{itemize}
    
        \item Small p functions means the arguments can be propositions or numerical values
            \begin{itemize}
                \item[] $p(x|y)$ or $p(A|B)$ or $p(x|B)$
            \end{itemize}
    
        \item Should be noted that this book only applies to finite sets of propositions
        \item \textit{Kernel}: This means something different in probability theory.\ Form of the PDF in which all factors which are not functions of any of the variables in the domain are omitted.
    \end{itemize}

\subsection{General Equations}
    \begin{itemize}
        \item Binomial coefficient
            \begin{equation*}
                \binom{n}{k} = \frac{n!}{k!(n-k)!}
            \end{equation*}
        \item Fourier transform:
            \begin{equation*}
                \mathscr{F}_{i}(\alpha) = \infint \mathrm{d}x~f_{i}(x)e^{i{\alpha}x} \hspace{1cm} f_{i}(x) = \frac{1}{2\pi} \infint \mathrm{d}{\alpha}~\mathscr{F}_{i}(\alpha)e^{-i{\alpha}x}
            \end{equation*}
        \item Power series:
            \begin{equation*}
                \sum^{\infty}_{n=0} a_{n}{(x-c)}^n = a_0 + a_{1}{(x-c)}^1 + a_{2}{(x-c)}^2 + \ldots
            \end{equation*}
            with special cases, such as for exponentials:
            \begin{equation*}
                e^x = \sum^{\infty}_{n=0} \frac{x^n}{n!} = 1 + \frac{x^2}{2!} + \frac{x^3}{3!} + \ldots
            \end{equation*}
        \item Stirling's approximation:
            \begin{equation*}
                \log (n!) \sim n \log (n) - n + \log \sqrt{2 \pi n} + O \left( \frac{1}{n} \right)
            \end{equation*}
        \item Gaussian error function:
            \begin{equation*} 
                \text{erf}(x) = \frac{1}{\sqrt{\pi}} \int^{x}_{-x} e^{-t^2} \text{d}t = \frac{2}{\sqrt{\pi}} \int^{x}_{0} e^{-t^2} \text{d}t
            \end{equation*}
        \item Poisson distribution (probability that \(n\) counts will appear in one timestep):\
            \begin{equation*} 
                p(n|l) = \frac{l^n}{n!}e^{-l} \hspace{1cm} n = 1, 2, \ldots
            \end{equation*}
            where \(l\) is the sampling expectation value of \(n\), \(\langle n \rangle = l\)
        \item Cauchy--Schwarz Inequality:
            \begin{equation*} 
                {|\langle \hat{u}, \hat{v} \rangle|}^2 \leq \langle \hat{u}, \hat{u} \rangle \cdot \langle \hat{v}, \hat{v} \rangle
            \end{equation*}
            where \(\langle \ldots, \ldots \rangle\) denotes an inner product.\ So, it states that the square of the inner product of two vectors, \(\hat{u}\) and \(\hat{v}\), is less than or equal to the dot product of their inner products.\\
            a more common notation for probability theory is
            \begin{equation*} 
                {|\langle XY \rangle|}^2 \leq {\langle X^2 \rangle} {\langle Y^2 \rangle}
            \end{equation*}
            where \(X\) and \(Y\) are random variables
        \item Laplace transform (takes a function of a real variable \(t\), usually time, to a function of a complex variable \(s\), usually frequency):
            \begin{equation*} 
                F(s) = \int^{\infty}_{0} \text{d}t~f(t)e^{-st}
            \end{equation*}
            where \(s\) is a complex number frequency parameter, \(s = \sigma + i\omega\), with \(\sigma\) and \(\omega\) being real numbers
    \end{itemize}

\end{document}
