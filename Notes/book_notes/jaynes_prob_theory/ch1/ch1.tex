\documentclass[../jaynes_prob_theory_notes.tex]{subfiles}
%\usepackage[margin=1in]{geometry}
%\usepackage{amsmath}

\begin{document}
\section{Boolean Algebra}
    \begin{itemize}
        \item AB -- logical product or conjunction
            \begin{itemize}
            \item both A and B are true, order does not matter
            \end{itemize}
        \item A + B -- logical sum or disjunction
            \begin{itemize}
            \item at least one of A or B is true, order does not matter
            \end{itemize}
        \item if A or B is true iff the other is true, they both have the same truth value
            \begin{itemize}
                \item does not matter how it is established that they have the same truth value
                \item leads to the most primative axiom of plausible reasoning: two propositions with the same truth value are equally plausible
            \end{itemize}
    \end{itemize}
    
    \subsection{Trivial identities of Boolean algebra}
    \begin{itemize}
        \item[] Idempotence:  
        $AA = A$
        $A + A = A$
    
        \item[] Commutativity:
            $AB = BA$
            $A + B = B + A$
        
        \item[] Associativity:
            $A(BC) = B(AC) = ABC$
            $A + (B + C) = (A + B) + C = A + B + C$
        
        \item[] Distributivity:
            $A(B+C) = AB + BC$
            $A + (BC) = (A + B)(A + C)$
        
        \item[] Duality:
            If $C + AB$, then $\bar{C} = \bar{A} + \bar{B}$
            If $D = A + B$, then $\bar{D} = \bar{A}\bar{B}$
          
        \item these trivial identities can be used to prove more important relations
        \item $A \Rightarrow B$
            \begin{itemize}
                \item A implies B, does not assert that either A or B is true
                \item same thing as $A\bar{B}$ is false
                \item if A is false, it says nothing about B, and vice versa
                \item means A and AB have the same truth value
            \end{itemize}
        \item the three operations, conjunction, disjunction, and negation (the bar over the letter) are adequate to generate all logic functions of a single proposition
    \end{itemize}

    \begin{itemize}
        \item Conditional probability: $A|B$
            \begin{itemize}
                \item the conditional probabiltiy that A is true given that B is true
                \item can use all the above identities
            \end{itemize}
    \end{itemize}
\end{document}
